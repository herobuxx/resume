\documentclass[11pt,a4paper]{moderncv}
\moderncvtheme{classic}
\usepackage[utf8]{inputenc}
\usepackage[scale=0.8]{geometry}

% Personal data
\firstname{Sam}
\familyname{Protsenko}
\title{Software Engineer}
\address{Ukraine, Kyiv}{}
\mobile{+38 097 5277313}
\email{joe.skb7@gmail.com}
\photo[64pt][0.4pt]{photo}
\social[github]{joe-skb7}
\social[linkedin]{sam-protsenko-491b311b}

\definecolor{web}{rgb}{0.2,0.2,0.2}

% to show numerical labels in the bibliography; only useful if you make citations in your resume
\makeatletter
\renewcommand*{\bibliographyitemlabel}{\@biblabel{\arabic{enumiv}}}
\makeatother

% Content
\begin{document}

% Remove "Bibliography" title in "thebibliography" block
\renewcommand*{\bibliographyhead}[1]{}

\maketitle

\section{Summary}

Software engineer with more than 10 years of professional experience.
Main area of expertise is Linux kernel, bootloaders, firmwares and system level
user space Embedded programming on ARM-based devices. For the last 5 years has
been contributing to open-source projects and involved with upstreaming
activity. Has prior background in cross-platform applications development in
C++/Qt, algorithms and electronics. Active member of open-source community,
mentor, speaker.

\section{Experience}
\cventry{2018--2020}{Team Leader}{GlobalLogic}{Kyiv}{}
  {Supporting the \textsc{BeagleBoard X15} board and upstreaming work.
   \begin{itemize}
     \item Porting Android to a new board and upstreaming it with a small team:
           \cite{a}
     \item Adopting VPN from upstream kernel in AOSP: \cite{b}
     \item Implementing Android 10 boot sequence: \cite{c}
   \end{itemize}}
\cventry{2012--2018}{Software Engineer}{GlobalLogic}{Kyiv}{}
  {Platform development in Linux kernel and bootloaders, upstreaming.
   \begin{itemize}
     \item Implementing and supporting drivers: \cite{d}
     \item Maintaining boot sequence and fastboot in U-Boot: \cite{e}
     \item Board bring ups
     \item Bug fixing (drivers, device tree, bootloader)
     \item Migrating to new kernel versions
     \item Implementing XIP boot from NOR flash boot (automotive)
     \item System boot time optimizations (automotive)
   \end{itemize}
  }
\cventry{2009--2012}{Software Developer}{AlterEGO}{Kramatorsk}{}
  {Cross-platform software design and coding.}

\section{Education}
\cventry{2004--2009}{Specialist degree}
  {Donbas State Mechanical Engineering Academy}{Kramatorsk}{}
  {\textit{Speciality}: Automation of technological processes}

\section{Computer skills}
\cvitem{Languages}{C, C++, Bash}
\cvitem{Projects}{Linux kernel, U-Boot, Android}
\cvitem{Architectures}{ARM32, AArch64, x86, STM32 (Cortex-M), MSP430}

\section{Languages}
\cvlanguage{Ukrainian}{Fluent}{My native language}
\cvlanguage{Russian}{Fluent}{My native language}
\cvlanguage{English}{Upper-intermediate}{Speaking, reading and writing}

\pagebreak

\section{Other Activities}

\cvitem{2011--Today}{Maintainer of ``INSTEAD'' Debian package: \cite{f}}
\cvitem{2015--Today}{Active member on StackOverflow (>10k reputation): \cite{g}}
\cvitem{2019}{Organizer of ``Root Linux Conference'' in Kyiv: \cite{h}}
\cvitem{2018--Today}{Creator and instructor of Linux kernel course: \cite{i}}
\cvitem{2017--2019}{Speaker at Linaro and GlobalLogic events}

\section{References}

\begin{thebibliography}{9}
\bibitem{a} \color{web} \url{https://android-review.googlesource.com/q/author:protsenko}
\bibitem{b} \color{web} \url{https://connect.linaro.org/resources/bkk19/bkk19-414/}
\bibitem{c} \color{web} \url{https://connect.linaro.org/resources/san19/san19-217/}
\bibitem{d} \color{web} \nolinkurl{https://git.kernel.org/pub/scm/linux/kernel/git/torvalds/linux.git/log/?qt=author&q=protsenko}
\bibitem{e} \color{web} \url{https://gitlab.denx.de/u-boot/u-boot/-/commits/master?search=protsenko}
\bibitem{f} \color{web} \url{http://packages.debian.org/sid/instead}
\bibitem{g} \color{web} \url{https://stackoverflow.com/users/3866447/sam-protsenko}
\bibitem{h} \color{web} \url{https://linux.globallogic.com/}
\bibitem{i} \color{web} \url{https://github.com/joe-skb7/kernel-lectures}
\end{thebibliography}

\end{document}
